\documentclass[11pt]{article}
\usepackage{amsmath,amssymb,amsthm}
\usepackage{algorithm}
\usepackage[noend]{algpseudocode} 
\usepackage{titlesec}
\usepackage{enumitem}
\usepackage[T2A,T1]{fontenc}
\usepackage{amsmath,amssymb,amsthm}
\usepackage{fancyhdr}
\usepackage{indentfirst}

%---enable russian----

\usepackage[utf8]{inputenc}
\usepackage[russian]{babel}

% PROBABILITY SYMBOLS
\newcommand*\PROB\Pr 
\DeclareMathOperator*{\EXPECT}{\mathbb{E}}


% Sets, Rngs, ets 
\newcommand{\N}{{{\mathbb N}}}
\newcommand{\Z}{{{\mathbb Z}}}
\newcommand{\R}{{{\mathbb R}}}
\newcommand{\Zp}{\ints_p} % Integers modulo p
\newcommand{\Zq}{\ints_q} % Integers modulo q
\newcommand{\Zn}{\ints_N} % Integers modulo N

% Landau 
\newcommand{\bigO}{\mathcal{O}}
\newcommand*{\OLandau}{\bigO}
\newcommand*{\WLandau}{\Omega}
\newcommand*{\xOLandau}{\widetilde{\OLandau}}
\newcommand*{\xWLandau}{\widetilde{\WLandau}}
\newcommand*{\TLandau}{\Theta}
\newcommand*{\xTLandau}{\widetilde{\TLandau}}
\newcommand{\smallo}{o} %technically, an omicron
\newcommand{\softO}{\widetilde{\bigO}}
\newcommand{\wLandau}{\omega}
\newcommand{\negl}{\mathrm{negl}} 

% Misc
\newcommand{\eps}{\varepsilon}
\newcommand{\inprod}[1]{\left\langle #1 \right\rangle}


 
\newcommand{\handout}[5]{
  \noindent
  \begin{center}
  \framebox{
    \vbox{
      \hbox to 5.78in { {\bf Научно-исследовательская практика} \hfill #2 }
      \vspace{4mm}
      \hbox to 5.78in { {\Large \hfill #5  \hfill} }
      \vspace{2mm}
      \hbox to 5.78in { {\em #3 \hfill #4} }
    }
  }
  \end{center}
  \vspace*{4mm}
}

\newcommand{\lecture}[4]{\handout{#1}{#2}{#3}{Scribe: #4}{Система верстки LaTeX}}

\begin{document}

\lecture{}{Лето 2020}{}{Торсукова Инна}
\newpage
\renewcommand{\headrulewidth}{0pt}


Которая может быть сформулирована как

\[|x-\frac{p_n}{q_n}|<\frac{1}{2bq_{n}}.\]

С учётом предположения, что $a/b\neq p_n/q_n$, разность $bp_{n}-aq_{n}$ является ненулевым целым числом, откуда $1\leq|bp_{n}-aq_{n}|$. Мы можем сразу сделать вывод, что

\[\frac{1}{bq_{n}}\leq|\frac{bp_{n}-aq_{n}}{bq_{n}}|=|\frac{p_{n}}{q_{n}}-\frac{a}{b}|\leq|\frac{p_{n}}{q_{n}}-x|+|x-\frac{a}{b}|<\frac{1}{2bq_{n}}+\frac{1}{2b^{2}}.\]
	
Это создает противоречие $b<q_{n}$, завершая доказательство.

\section{Проблемы}
\noindent
\begin{enumerate}
	
\item Оцените каждую из следующих бесконечных простых цепных дробей:
\\
\begin{enumerate} 
	
	
\item $[\overline{2;3}]$
\\
\item  $[0;\overline{1,2,3}]$
\\
\item  $[2;\overline{1,2,1}]$ 
\\
\item  $[1;2,\overline{3,1}]$
\\
\item  $[1;2,1,2,\overline{12}]$

\end{enumerate}
\item Докажите, что если иррациональное число $x>1$ представимо в виде бесконечной цепной дроби $[a_0;a_1,a_2,...]$, то $1/x$ имеет разложение $[0;a_0;a_1,a_2,...]$. Используйте этот факт, чтобы найти значение $[0;1,1,1,...]=[0,\overline{1}]$.

\item Оцените $[1;2,\overline{1}]$ и $[1;2,3,\overline{1}]$.

\item Представьте каждое иррациональное число в виде бесконечной цепной дроби:
\begin{enumerate} 				
\item $\sqrt{5}$\\
\item $\sqrt{7}$\\
\item $\frac{1+\sqrt{13}}{2}$\\
\item $\frac{5+\sqrt{37}}{4}$\\
\item $\frac{11+\sqrt{30}}{13}$
\end{enumerate}	
			 
\item
\begin{enumerate}
 \item Для любого натурального числа $n$ покажите, что $\sqrt{n^{2}+1}=[n;\overline{2n}]$, $\sqrt{n^{2}+2}=[n;\overline{n,2n}]$ и $\sqrt{n^{2}+2n}=[n;\overline{1,2n}]$.\\ \textit{Подсказка}: Обратите внимание на это \[ n+\sqrt{n^{2}+1}=2n+(\sqrt{n^{2}+1}-n)=2n+\frac{1}{n+\sqrt{n^{2}+1}} .  \] 
\item Используйте часть $(a)$, чтобы представить $ \sqrt{2}, \sqrt{3}, \sqrt{15}$ и $\sqrt{37}$ в виде бесконечных цепных дробей.
\end{enumerate}
			
\item Среди конвергентов $\sqrt{15}$ найдите рациональное число, которое будет приближённо равно $\sqrt{15}$ с точностью до четвёртого знака после запятой. 

\item 
\begin{enumerate}
\item Найдите рациональное приближение к $e=[2;1,2,1,1,4,1,1,6,...]$, что соответствует $4$ десятичным разрядам.
\item Если $a$ и $b$ натуральные числа, покажите, что из неравенства $0<a/b<87/32$ следует, что $b\geq39$.
\end{enumerate}

\item Докажите, что из любых двух последовательных сходимостей иррационального числа $x$ хотя бы одно, $a/b$ удовлетворяет неравенству \[|x-\frac{a}{b}|<\frac{1}{2b^{2}}\] \textit{Подсказка}: Поскольку $x$ лежит между любыми двумя последовательными конвергентами, \[\frac{1}{q_{n}q_{n+1}}=|\frac{p_{n+1}}{q_{n+1}}-\frac{p_{n}}{q_{n}}|=|x-\frac{p_{n+1}}{q_{n+1}}|+|x-\frac{p_{n}}{q_{n}}|.\] Теперь рассуждайте от противного.

\item Учитывая бесконечную непрерывную дробь $[1;3,1,5,1,7,1,9,...],$ найдите наилучшее рациональное приближение $a/b$ с 
\begin{enumerate}
\item знаменателем $b<25;$ 
\item знаменателем $b<225.$
\end{enumerate}
	
\item Сначала покажите, что $|(1+\sqrt{10})/3-18/13|<1/(2\cdot13^{2})$; и затем убедитесь, что $18/13$ сходится к $(1+\sqrt{10}/3)$.

\item В известной теореме А. Гурвица(1891) говорится, что для любого иррационального числа $x$ существует бесконечно много рациональных чисел $a/b$ таких, что \[|x-\frac{a}{b}|<\frac{1}{\sqrt{5}b^{2}}.\] \\ Взяв $x=\pi$, получим три рациональных числа, удовлетворяющих этому неравенству.

\item Предположим,что представление непрерывной дроби для иррационального числа $x$ в конечном итоге становится периодическим. Имитируйте метод, использованный в примере 13-4,  чтобы доказать, что $x$ имеет форму $r+s\sqrt{d}$, где $r$ и $s\neq0$ - рациональные числа, а $d>0$ - целое число, не являющееся квадратом.

\item Пусть $x$ - иррациональное число с конвергентами $p_{n}/q_{n}$. Для всех $n\geq0$, убедитесь, что 
\begin{enumerate}
\item $ 1/2q_{n}q_{n+1}<|x-p_{n}/q_{n}|<1/q_{n}q_{n+1}$; 
\item конвергенты стремятся к $x$ в том смысле, что \[|x-\frac{p_{n}}{q_{n}}|<|x-\frac{p_{n-1}}{q_{n-1}}|.\]
\end{enumerate} 
\textit{Подсказка}: Перепишите отношение \[x=\frac{x_{n+1}q_{n}+p_{n-1}}{x_{n+1}q_{n}+q_{n-1}}\] 
\\ как $x_{n+1}(xq_{n}-p_{n})=-q_{n-1}(x-p_{n-1}/q_{n-1}).$

 
\end{enumerate}

\newpage
		
\section{Уравнение Пелла}

То немногое, что Ферма предпринял для обнародования своих открытий, вылилось в вызов другим математикам. Возможно, он надеялся таким образом убедить их, что его новый стиль теории чисел заслуживает продолжения. В январе 1657 Ферма предложил европейскому математическому сообществу подумать, вероятно, в первую очередь о Джоне Уоллисе, самом известном английском практикующем до Ньютона, пара проблем: 
\begin{enumerate}  
\item Найдите куб, который при сложении с его собственными делителями становится квадратом; например, $7^{3}+(1+7+7^{2})=20^{2}$.
\item Найдите квадрат, который при сложении с его собственными делителями становится кубом.
\end{enumerate}
Услышав о конкурсе, любимый корреспондент Ферма, Бернар Френикль де Бесси, быстро предоставил ряд ответов на первую проблему; типичным из них является $(2\cdot3\cdot5\cdot13\cdot41\cdot47)^{3}$, который при увеличении на сумму его собственных делителей становится $(2^{7}\cdot3^{2}\cdot5^{2}\cdot7\cdot13\cdot17\cdot29)^{2}$. В то время как Френикль продвигался к решениям в еще больших составных числах, Уоллис отклонил проблемы как не стоящие его усилий, написав, "Какими бы ни были подробности этого вопроса, он находится мне слишком загруженным многочисленными вычислениями, чтобы я мог сразу обратить на него свое внимание, но я могу дать в этот момент такой ответ: число $1$ само по себе удовлетворяет обоим требованиям." Едва скрывая свое разочарование, Френикль выразил удивление, что такой опытный математик, как Уоллис, мог дать только тривиальный ответ, когда, учитывая рост Ферма, он должен был почувствовать большую глубину проблемы.\\
Действительно, интерес Ферма заключался в общих методах, а не в изнурительных вычислениях отдельных случаев. И Френикль, и Уоллис упустили из виду теоретический аспект, который должен был выявить проблемы-вызовы при тщательном анализе. Хотя формулировка была не совсем точной, кажется очевидным, что Ферма намеревался решить первый из своих запросов для кубов простых чисел. Иначе говоря, задача требовала нахождения всех интегральных решений уравнения \[1+x+x^{2}+x^{3}=y^{2},\] 
или эквивалентно \[(1+x)(1+x^{2})=y^{2},\] 
где $x$ - нечетное целое число. Поскольку $2$ - единственное простое число, которое делит оба множителя в левой части этого уравнения, оно может быть записано как \[ ab=(\frac{y}{2})^{2},\] \[\text{НОД}(a,b)=1\]  
Но если произведение двух взаимо простых чисел является идеальным квадратом, то каждое из них должно быть квадратом; следовательно, $a=u^{2}, b=v^{2}$ для некоторых $u$ и $v$, так что \[1+x=2a=2u^{2},\] \[ 1+x^{2}=2b=2v^{2}.\] 
Это означает, что любое целое число $x$, удовлетворяющее условиям первой проблемы Ферма, должно быть решением пары уравнений \[x=2u^{2}-1,\] \[ x^{2}=2v^{2}-1,\] 
второе - частный случай уравнения $x^{2}=dy^{2}\pm1.$ \\
В феврале 1647 года Ферма бросил свой второй вызов, непосредственно связанный с теоретическим вопросом: Найти число $y$, которое сделает $dy^{2}+1$ идеальным квадратом, где $d$ - положительное целое число, которое не является квадратом; например, $3\cdot1^{2}+1=2^{2}$ и $5\cdot4^{2}+1=9^{2}.$ Если, сказал Ферма, общее правило не может быть получено, найдите наименьшие значения $y$, которые удовлетворят уравнениям $61y^{2}+1=x^{2}$ или $109y^{2}+1=x^{2}.$ Френикль приступил к вычислению наименьших положительных решений $x^{2}-dy^{2}=1$ для всех доступных значений $d$ до $150$ и предложил Уоллису расширить таблицу до $d=200$ или, по крайней мере, решить $x^{2}-151y^{2}=1$ и $x^{2}-313y^{2}=1$, намекая, что второе уравнение может оказаться за пределами возможностей Уоллиса. В ответ патрон Уоллиса, лорд Уильям Брункер из Ирландии заявил, что ему потребовался всего час или около того, чтобы обнаружить, что \[(126862368)^{2}-313(7170685)^{2}=-1\] 
и поэтому $y=2\cdot7170685\cdot126862368$ дает желаемое решение для $x^{2}-313y^{2}=1;$ Уоллис решил другой конкретный случай, поставив \[(1728148040)^{2}-151(140634693)^{2}=1.\] Размер этих чисел по сравнению с теми, которые возникают из других значений $d$, говорит о том, что Ферма обладал полным решением задачи, но это никогда не раскрывалось (позже он утверждал, что его метод бесконечного спуска был успешно использован, чтобы показать существование бесконечности решений $x^{2}-dy^{2}=1$). Брункер,


\end{document}

